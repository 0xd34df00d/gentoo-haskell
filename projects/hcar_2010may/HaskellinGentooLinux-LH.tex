\begin{hcarentry}[updated]{Haskell in Gentoo Linux}
\label{gentoo}
\report{Lennart Kolmodin}%11/09
\makeheader

Gentoo Linux currently supports GHC 6.10.4, including the
latest Haskell Platform~\cref{haskell-platform} for x86 and amd64.
For previous GHC versions we have binaries available for alpha, amd64, hppa,
ia64, sparc, and x86.

Browse the packages in portage at 
\url{http://packages.gentoo.org/category/dev-haskell?full\_cat}.

The GHC architecture/version matrix is available at
\url{http://packages.gentoo.org/package/dev-lang/ghc}.

Please report problems in the normal Gentoo bug tracker
at \url{bugs.gentoo.org}.

There is also a Haskell overlay providing another 300 packages. Thanks to
the haskell developers using Cabal and Hackage~\cref{hackagedb}, we have been
able to write a tool called ``hackport'' (initiated by Henning G\"unther) to
generate Gentoo packages that rarely need much tweaking.

Read about the Gentoo Haskell Overlay at
\url{http://haskell.org/haskellwiki/Gentoo}. Using
Darcs~\cref{darcs}, it is easy to keep updated and send patches.
It is also available via the Gentoo overlay manager ``layman''.
If you choose to use the overlay, then problems should be
reported on
IRC (\verb+#gentoo-haskell+ on freenode), where we coordinate
development, or via email \email{haskell@gentoo.org}.

Through recent efforts we have devoped a tool called ``haskell-updater''
\url{http://www.haskell.org/haskellwiki/Gentoo#haskell-updater}.
It helps the user when upgrading GHC versions, fixes
breakages from library upgrades, etc.

As always we are happy to get help hacking on the Gentoo Haskell
framework, hackport, writing ebuilds, and supporting users. Please contact
us on IRC or email if you are interested!
\end{hcarentry}
